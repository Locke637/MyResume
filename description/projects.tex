\resheading{学术背景}

\ressubsingleline{APRIL机器人实验室}{浙江大学智能控制研究所}{2019.09 -- 至今}
{\small
  \begin{itemize}
    \item 研究方向主要包括:多智能体强化学习,模仿学习,分层强化学习
    \item 提出了一种新的去中心化、基于局部观测的多智能体强化学习算法来解决多智能体编队(MAiF)任务。该算法使用分层强化学习结构将多目标任务分解为互相解耦的任务,同时计算理论权重使每个任务的奖励对最终强化学习值函数具有同等影响。实验结果表明我们的算法对于地图大小变化具有很好的迁移性。
    \item 为了改进不完全信息自博弈下的强化学习的训练过程,在策略梯度方法中加入了一个综合全局信息的评论家,形成了一个自博弈演员-评论家(SPAC)方法,用于训练智能体玩电脑游戏。在竞争和合作等多种博弈任务下的结果显示,该方法要优于基线算法的效果。
    \item 提出了集中训练和分布式执行可适应动态智能体数目的多智能体强化学习算法。使用注意力机制来选择若干队长并建立动态数量的团队,。
    \item 提出了一种多智能体全局价值函数分解方法,其既考虑了智能体的独立行动的回报,也考虑了与附近其他智能体合作的回报以解决QMIX类方法的单调性问题。此外,我们提出了一种贪婪动作搜索方法,该方法可以改进探索,并且使智能体的策略不受附近智能体数目的变化或动作顺序变化的影响。实验结果表明,整体的方法在三个具有挑战性的MAgent任务中实现了显着的性能提升,并且可以处理未见过的合作场景。
    \item 提出了在非时间对齐的环境中仅从观测值中模仿学习的方法,其采用分层强化学习结构从专家轨迹的观察值中动态选择可行的子目标。同时可以通过使用设计的奖励结构针对不同类型的任务学习对应类型的策略。此外还提出了三种不同的方法来提高层次结构中的样本利用效率。实验结果表明,整体的方法的性能和学习效率都有所提高。
  \end{itemize}
}

\ressubsingleline{西澳大学}{}{2018.07 -- 2018.09}
{\small
  \begin{itemize}
    \item 参与西澳大学机器人无人驾驶系统开发项目。该系统的架构由车道检测、交通标志识别、停车、通信和人机界面组成。该项目中考虑了几种场景,包括正常巡线模式和停车模式。在正常巡线模式下,无人驾驶小车可以自动沿着车道行驶,并实时检测前方各种交通标志并做出相应反应,例如一旦检测到停车标志就会进入停车模式并搜索停车场停车。
  \end{itemize}
}


\resheading{学术论文}
\begin{itemize}[leftmargin=*]
  {\small

  \item
  \textbf{S. Liu}, L. Wen, J. Cui, X. Yang and Y. Liu, "Moving Forward in Formation: A Decentralized Hierarchical Learning Approach to Multi-Agent Moving Together". (IROS 2021 Accepted)
  \item
  \textbf{S. Liu}, J. Cao, Y. Wang, W. Chen and Y. Liu, "Self-Play Reinforcement Learning with Comprehensive Critic in Computer Games." (Neurocomputing Accepted)
  \item
  W. Liu, \textbf{S. Liu}, J. Cao, Q. Wang, X. Lang and Y. Liu, "Learning Communication for Cooperation in Dynamic Agent-Number Environment." (IEEE/ASME Transactions on Mechatronics Accepted)
  \item 
  S. Sun, J. Zheng, Z. Qiao, \textbf{S. Liu}, Z. Lin and T. Bräunl, "The Architecture of a Driverless Robot Car Based on EyeBot System." (Journal of Physics: Conference Series Accepted)
  \item
  \textbf{S. Liu}, W. Liu, W. Chen, J. Cao and Y. Liu, "Learning Multi-Agent Cooperation via Considering Actions of Teammates." (NeurIPS 2021, Under Review)}
  \item
  \textbf{S. Liu}, J. Cao, W. Chen, L. Wen and Y. Liu, "HILONet: Hierarchical Imitation Learning from Non-Aligned Observations". (IEEE Transactions on SMC: Systems, Under Review)

\end{itemize}