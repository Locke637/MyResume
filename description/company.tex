\resheading{实习经历}
\begin{itemize}[leftmargin=*]
  \item
        \ressubsingleline{腾讯集团}{Robotics X 实习生}{2021.06 -- 2021.09}
        {\small
          \begin{itemize}
            \item 参与编写搭载在边缘计算平台上的一个机器人控制工具箱(Control Toolbox)。该工具箱完全使用C++编写,专门为求解机器人的动力学方程,以及基于模型的控制规划等对算力要求大、实时性要求高的优化问题设计。
            \item  目前应用在Ollie,Jamoca等腿足机器人上。
          \end{itemize}
        }
        % \item
        %       \ressubsingleline{嬴彻科技}{自动驾驶实习生}{2021.03 -- 2021.06}
        %       {\small
        %         \begin{itemize}
        %           \item 提出了一种基于深度学习的高速公路场景下的微观交通流驾驶模型,该模型相比于传统驾驶模型更接近人类司机驾驶习惯,因而更适合为自动驾驶算法提供交通流仿真。
        %           \item 设计了分层的驾驶模型神经网络,上层网络感知周围环境并进行当前帧的行为决策,下层网络根据上层行为决策输入和周围环境输出下一帧的控制信息。
        %           \item 使用真实公路行驶数据集进行训练,并在SUMO仿真中进行测试。结果显示,相比于传统的IDM驾驶模型,我们的驾驶模型在速度分布,变道轨迹等指标上均优于前者。
        %         \end{itemize}
        %       }
  \item
        \ressubsingleline{阿里巴巴集团}{Java实习开发工程师}{2018.07 -- 2018.09}
        {\small
          \begin{itemize}
            \item 参与设计了天猫灰度环境平台的迭代升级,该平台是天猫双十一购物狂欢节测试上线的重要环节
            \item 主要解决了集团应用的灰度发布繁琐的问题,提供了一个自动化的灰度环境租赁平台。项目累计服务集团包括飞猪,菜鸟,盒马生鲜,天猫精灵等十余个事业群,覆盖超过 300 个应用。
            \item 天猫技术部优秀项目。集团百技实习生培训优秀个人。
          \end{itemize}
        }
        % \ressubsingleline{网易互娱}{游戏设计实习生}{2018.09 -- 2018.12}
        % {\small
        %   \begin{itemize}
        %     \item 参与设计手游明日之后的多个关卡和数值。分析了刺客信条奥德赛的游戏关卡地图和英雄伤害数值。
        %   \end{itemize}
        % }
\end{itemize}