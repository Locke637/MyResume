\resheading{Academic Research}

\ressubsingleline{\textbf{\href{https://april.zju.edu.cn/}{APRIL Laboratory}}}{Institute of Cyber Systems and Control , ZJU}{2019.09 -- Present}
{\small
  \begin{itemize}
    \item Research interests: Path/motion planning, Multi-agent systems, Mobile robots
    \item Propose a hierarchical search algorithm for the multi-agent path finding problem (CL-MAPF) based on Ackermann model kinodynamic constraints. Its high level uses a body conflict search tree to consider the collision problem between vehicles, and its low level introduces a spatio-temporal hybrid A* algorithm as a single-body path planner. Our algorithm is verified in both simulations and real tests.
    \item Propose a decentralized, locally observed reinforcement learning algorithm to solve multi-agent in formation (MAiF) tasks. The algorithm uses a hierarchical reinforcement learning structure to decompose the multi-objective task into mutually decoupled tasks. Experimental results show that our algorithm has good mobility for map size variations.

    \item Propose an unmanned boat trajectory planning method that accurately generates smooth and collision avoidance trajectories based on dynamics constraints. The trajectory planning problem is decoupled into front-end feasible path search and back-end kinematic dynamics trajectory optimization. Finally, the method is verified to meet our expected trajectory in a simulation environment.
          % \item 为一型30吨无人艇设计了智能搜救系统。主要功能包括基于无人艇现有的多源传感器信息进行数据融合,根据任务详情进行路径规划并具备实时避障的能力;使用深度学习以可见光/全景/红外相机的图像输入为主,雷达探测为辅,对可疑目标进行识别并进行目标重要度排序。
    \item Proposes a learning-based microscopic traffic flow driving model for highway scenarios, which adopts a hierarchical neural network framework. It is closer to the driving habits of human drivers than conventional driving models and thus more suitable for providing traffic flow simulation for autonomous driving algorithms. The models are trained using a real highway driving dataset and tested in SUMO simulation.
  \end{itemize}
}

\ressubsingleline{\textbf{\href{https://www.youtube.com/channel/UCaxpfkJ9pvyCnQynVlQLE1A}{ZJUNlict RoboCup Team}}}{State Key Laboratory of Control Technology , ZJU}{2016.09 -- 2019.09}
{\small
  \begin{itemize}
    \item Core member of software group. Focused on computer vision and AI strategy.
    \item Rewrite vision module with a friendly interface and enable AI to acquire more information from the field. Applied a brand-new position filter to our 150k+ LOC program to get more stable and accurate position using binocular vision. Use collision detect algorithm to monitor the objects collision between robots and ball.
  \end{itemize}
}


\resheading{Papers}
\begin{itemize}[leftmargin=*]
  {\small
  \item
        \textbf{L. Wen}, Z. Zhang, J. Yan, X. Zhao and Y. Liu, "Hetero-MAPF: An Efficient Multi-Agent Path Finding Approach
        for Heterogeneous Mobile Robots", (Submitted to IROS 2021)
  \item
        S. Liu,\textbf{ L. Wen}, J. Cui, X. Yang and Y. Liu, "Moving Forward in Formation: A Decentralized Hierarchical Learning Approach to Multi-Agent Moving Together", (IEEE Transactions on Mechatronics, Under Review)

  \item
        \textbf{L. Wen}, J. Yan, H. Wu, X. Yang , J. Wang and Y. Liu, "TorPeDo: A Smooth and Safe Trajectory Planner for Differential Unmanned Surface Vehicle", (IEEE Transactions on SMC: Systems, Under Review)

  \item
        \textbf{L. Wen}, J. Yan, X. Yang, Y. Liu and Y. Gu, "Collision-free Trajectory Planning for Autonomous Surface Vehicle," 2020 IEEE/ASME International Conference on Advanced Intelligent Mechatronics (AIM), Boston, MA, USA, 2020, pp. 1098-1105
        }
\end{itemize}